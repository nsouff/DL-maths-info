\documentclass[td4.tex]{subfiles}
\begin{document}
\section{Exercice 5}
\subsection{Question 1}
$
\phi(e_1,e_1) = \phi(e_2, e_2) = \frac{\pi}{2} \\
\phi(e_1,e_2) = 0 \\
\phi(e_1,e_3) = 0 \\
\phi(e_2,e_3) = 2
$
De plus $\phi$ est clairement symétrique ainsi, soit $M$ la matrice de $\phi$ dans la base $(e_1, e_2, e_3)$, on a :
$$ M =
\begin{pmatrix}
  \frac{\pi}{2} & 0 & 0 \\
  0 & \frac{\pi}{2} & 2 \\
  0 & 2 & \pi
\end{pmatrix}
$$

\subsection{Question 2}
Soit $f = \begin{pmatrix} a \\ b \\ c \end{pmatrix}\text{dans la base}(e_1,e_2,e_3),~\phi$ orthogonale à $e_1$ et $e_3$, alors:
$0 = \phi(f,e_1) = \begin{pmatrix}a & b & c\end{pmatrix}\begin{pmatrix} \frac{\pi}{2} \\ 0 \\ 0 \end{pmatrix} = a\frac{\pi}{2} \implies a = 0$\\
Et \\
$0 = \phi(f, e_3) = \begin{pmatrix}0 & b & c\end{pmatrix} \begin{pmatrix}0 \\ 2 \\ \pi \end{pmatrix} = 2b + c\pi \implies b = -\frac{\pi}{2}c$ \\
Donc $f \in F \iff f = -\lambda \frac{\pi}{2} e_2 + \lambda$, ainsi une base de $F$ est $(sin(x), 1)$

\end{document}
