\documentclass[td7.tex]{subfiles}

\begin{document}

\section*{Exercice 3}
On doit associer chaque écrou à sa vis. Pour cela, on va trier simultanément les vis et les écrous en s'inspirant du \emph{tri rapide}.

\begin{enumerate}
\item S'il y a une vis et un écrou alors c'est terminé.
\item On partitionne les vis à l'aide d'un pivot $e_i$ choisi dans les écrous.
  On mémorise durant cette étape l'unique vis $v_i$ associée à $e_i$.
\item On partitionne les écrous à l'aide du pivot $v_i$.
\item \`{A} la fin des étapes 1 et 2, les deux partitions coïncident. De plus, on a associé $e_i$ à $v_i$.
  On reprend alors à l'étape 1 pour les vis inférieurs à $e_i$ et les écrous inférieurs à $v_i$, de même pour les parties supérieures.
\end{enumerate}

\inputminted[mathescape, firstline=10, lastline=31]{python}{vis_ecrou.py}

\end{document}