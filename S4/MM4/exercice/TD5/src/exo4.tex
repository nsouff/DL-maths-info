\documentclass[td5.tex]{subfiles}



\begin{document}

\section{Exercice 4}
\subsection*{Question 1}

  \textbf{Convergence simple:}\\
  $\forall n \in \mathbb{N},~f_n(1) = 0$ \\
  $\forall x \ne 1$ clairement $\lim\limits_{n \to +\infty}f_n(x) = 0$\\
  \textbf{Convergence uniforme:}\\
  On suppose $\varepsilon < 1$\\
  Si $1 - \varepsilon < x$, on a $\forall n \in \mathbb{N}$:\\
  $$x^n(1-x)\leq 1 - x < \varepsilon$$\\
  Si $x \leq 1 - \varepsilon$, on a :
  $$x^n(1-x) \leq (1-\varepsilon)^n(1-x) \leq (1-\varepsilon)^n$$\\
  Si on prend donc $n_0$ tel que $(1-\varepsilon)^{n_0} < \varepsilon$ on a:
  $x^n(1-x) < \varepsilon,~\forall n \geq n_0 $\\
  Ainsi $\forall n \geq n_0,~\forall x \in [0, 1],~f_n(x) < \varepsilon$

\subsection{Question 2}
$g_n(x) = x^n sin(\pi x) = x^n sin(\pi - \pi x) = x^n sin(\pi(1-x))$\\
De plus $\forall x \geq 0,~ sin(x) \leq x$ Donc: \\
\begin{align*}
  &sin(\pi (1-x)) \leq \pi (1-x)  \\
  \implies ~&x^n sin (\pi(1-x)) \leq \pi x^n (1-x),~ \forall x \in [0, 1] \\
  \implies ~&0 \leq g_n(x) \leq f_n(x) \forall x \in [0, 1]\\
\end{align*}
On sait que $f_n \xrightarrow{u} 0$, alors \\
$\forall \varepsilon > 0 \exists n_0$ tel que $\forall n \geq n_0,~\forall x |f_n(x)| < \varepsilon$
Ainsi $\forall x, ~\forall n \geq n_0$\\
$|g_n(x)| \leq |f_n(x)| < \varepsilon$
Donc $g_n \xrightarrow{u} 0$

\end{document}
