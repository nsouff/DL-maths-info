\documentclass[td5.tex]{subfiles}


\begin{document}
\section{Exercice 7}
\subsection{Question 1}
$f_n(x) = e^{-nx}sin(2nx)$
$x > 0,~ |f_n(x)| = |e^{-nx}||sin(2nx)| \leq e^{-nx}$ et $\lim\limits_{n \to + \infty} e^{-nx} = 0$ \\
Donc $f_n \xrightarrow{s} 0$
\textbf{Convergence uniforme}
Supposons que $f_n \xrightarrow{u} 0$, alors soit $\varepsilon < e^{-\frac{\pi}{4}},~\exists n_0 \in \mathbb{N}$tel que \\
$\forall n \geq n_0,~\forall x \in [0, +\infty[ |f_n(x)| < \varepsilon$,
Or, prenons $x = \frac{\pi}{4n_0}$, on a alors: \\
\begin{align*}
  &|e^{-n_0 \frac{\pi}{4n_0}} sin(2n_0 \frac{\pi}{4n_0})| \\
  = &e^{- \frac{\pi}{4}} sin(\frac{\pi}{2})\\
  = &e^{- \frac{\pi}{4}} \geq \varepsilon \\
\end{align*}
Donc $f_n \not\xrightarrow{u} 0 sur ]0, +\infty[$ et donc également sur $[0, +\infty[$.
Mais $f_n \xrightarrow{u} 0$ sur $[a, +\infty[,~a > 0$
car soit $\varepsilon > 0,~|f_n(x)| \leq e^{-nx} \leq {-na}$ \\
Donc si on prend $n_0$ tel que $e^{-n_0 a} < \varepsilon$,
on a bien $|f_n(x)| < \varepsilon,~\forall n\geq n_0,~\forall x \in [a, +\infty[$

\subsection{Question 2}
Clairement
$f_n(x) \xrightarrow{s} \left\{
  \begin{array}{ll}
      1 & \mbox{si } x = 0 \\
      0 & \mbox{sinon}
    \end{array}
  \right.
$ \\
Donc $f_n$ converge simplement vers un fonction non continue alors que $f_n$
est continue $\forall n \in \mathbb{N}$. \\
Il vient, $f_n$ ne converge pas uniformément

\end{document}
