\documentclass[td7.tex]{subfiles}

\begin{document}

\section*{Exercice 2}
\inputminted{python}{triRapide.py}

\subsection*{Question 1}
\begin{itemize}
\item Le pire cas arrive lorsque le pivot sépare le tableau de taille $n$ en une partie vide et en une partie de taille $n-1$. La taille de la pile est alors en $O(n)$.
\item Le meilleur cas arrive lorsque le pivot choisi est la médiane.
  Le tableau est donc séparer en deux parties de même taille. La taille de la pile est alors en $O(\log(n))$.
\end{itemize}


\subsection*{Question 2}
\begin{itemize}
\item Sur l'entrée $[1, \dots, n]$, la pile atteint une hauteur de 1. En effet, la récursion est appelée sur la partie gauche vide.
\item Sur l'entrée $[n, \dots, 1]$, la pile atteint une hauteur en $O(n)$. La récursion est appelée sur la partie gauche de taille $n-1$. 
\end{itemize}

\subsection*{Question 3}
Il faut effectuer la récursion sur la partie la plus petite.
Le pire cas (taille de la pile) se produit lorsque les deux parties sont de même taille, la pile atteint alors une hauteur en $O(\log(n))$.

\subsection*{Question 4}
Le meilleur cas pour la hauteur de la pile se produit sur des entrées comme $[1, \dots, n]$ ou $[n, \dots, 1]$, la pile a une hauteur en $O(1)$.
Cela correspond à des appels récursifs sur des tableaux de taille inférieur à 1.

\end{document}
