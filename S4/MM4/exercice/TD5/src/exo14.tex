\documentclass[td5.tex]{subfiles}

\begin{document}

\section{Exercice 14}

\subsection*{Question 1}
Montrons que $\lim\limits_{k \to +\infty} f_k(x_k) = f(x)$. On a :
\begin{align*}
  \forall k \in \mathbb{N}~|f_k(x_k)-f(x)| &= |f_k(x_k)-f(x_k)+f(x_k)-f(x)|\\
                                           &\leq |f_k(x_k)-f(x_k)|+|f(x_k)-f(x)|
\end{align*}


$(f_k)_{k \geq 0}$ converge uniformément vers $f$ sur $I$ donc :
\begin{itemize}
\item $\exists n_0 \in \mathbb{N} ~ \forall k \in \mathbb{N} ~ k \geq n_0 \implies~|f_k(x_k) - f(x_k)| < \varepsilon$
\item $f$ est continue sur $I$ donc $\lim\limits_{k \to +\infty} f(x_k) = f(x)$.
\end{itemize}
Finalement $\exists n_1 \in \mathbb{N} ~ \forall k \in \mathbb{N}, ~ k \geq n_1 \implies~|f_k(x_k) - f(x)| < \varepsilon$


\subsection*{Question 2}
Il suffit de trouver une suite $(x_k)_{k \geq 0}$ d'élements de $I$ convergeant vers $x \in I$ et de montrer que $\lim\limits_{k \to +\infty} f_k(x_k) \neq f(x)$.


\end{document}