\documentclass[td7.tex]{subfiles}

\begin{document}

\section*{Exercice 4}
\subsection*{Question 1}
\inputminted[mathescape, firstline=31, lastline=39]{python}{problemeDrapeauHollandais.py}

\subsection*{Question 2}
\inputminted[mathescape, firstline=41, lastline=47]{python}{problemeDrapeauHollandais.py}

\subsection*{Question 3}
\inputminted[mathescape, firstline=49, lastline=61]{python}{problemeDrapeauHollandais.py}
Les variables bleu, rouge et blanc agissent comme des frontières :
\begin{itemize}
\item bleu est le premier indice après la zone bleue
\item blanc est le premier indice après la zone blanche
\item rouge est le premier indice avant la zone rouge
\end{itemize}

\subsection*{Question 4}
On peut considérer :

\begin{itemize}
\item la zone bleue comme les éléments inférieurs au pivot.
\item la zone blanche comme les éléments égaux au pivot.
\item la zone rouge comme les éléments supérieurs au pivot.
\end{itemize}
On effectue alors un appel récursif sur la zone bleue et sur la zone rouge.

\inputminted[firstline=63, lastline=88]{python}{problemeDrapeauHollandais.py}

\end{document}